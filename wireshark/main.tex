\documentclass[13pt, a4paper]{article}

\usepackage[utf8]{inputenc}
\usepackage[brazil]{babel}

\usepackage[centertags]{amsmath}
\usepackage[svgnames]{xcolor}
\usepackage[section]{placeins}

\usepackage[lmargin=0.1cm, rmargin=0.1cm, tmargin=0.1cm, bmargin=0.1cm]{geometry}

\usepackage{graphicx, amssymb, hyperref, color, ifxetex, ifluatex, libertine, listings, wrapfig, sectsty, pdfcomment, parskip}

\newcommand{\HRule}{\rule{\linewidth}{1pt}}

\sectionfont{\LARGE}
\subsectionfont{\Large}
\subsubsectionfont{\large}

% conditional for xetex or luatex
\newif\ifxetexorluatex
\ifxetex
  \xetexorluatextrue
\else
  \ifluatex
    \xetexorluatextrue
  \else
    \xetexorluatexfalse
  \fi
\fi
%
\ifxetexorluatex%
  \usepackage{fontspec}
  \usepackage{libertine} % or use \setmainfont to choose any font on your system
  \newfontfamily\quotefont[Ligatures=TeX]{Linux Libertine O} % selects Libertine as the quote font
\else
  \usepackage[utf8]{inputenc}
  \usepackage[T1]{fontenc}
  \usepackage{libertine} % or any other font package
  \newcommand*\quotefont{\fontfamily{LinuxLibertineT-LF}} % selects Libertine as the quote font
\fi

\newcommand*\quotesize{60} % if quote size changes, need a way to make shifts relative
% Make commands for the quotes
\newcommand*{\openquote}
   {\tikz[remember picture,overlay,xshift=-4ex,yshift=-2.5ex]
   \node (OQ) {\quotefont\fontsize{\quotesize}{\quotesize}\selectfont``};\kern0pt}

\newcommand*{\closequote}[1]
  {\tikz[remember picture,overlay,xshift=4ex,yshift={#1}]
   \node (CQ) {\quotefont\fontsize{\quotesize}{\quotesize}\selectfont''};}

% select a colour for the shading
\colorlet{shadecolor}{Azure}

\newcommand*\shadedauthorformat{\emph} % define format for the author argument

% Now a command to allow left, right and centre alignment of the author
\newcommand*\authoralign[1] {
  \if#1l
    \def\authorfill{}\def\quotefill{\hfill}
  \else
    \if#1r
      \def\authorfill{\hfill}\def\quotefill{}
    \else
      \if#1c
        \gdef\authorfill{\hfill}\def\quotefill{\hfill}
      \else\typeout{Invalid option}
      \fi
    \fi
  \fi
}


% wrap everything in its own environment which takes one argument (author) and one optional argument
% specifying the alignment [l, r or c]
\newenvironment{shadequote}[2][l]%
{\authoralign{#1}
\ifblank{#2}
   {\def\shadequoteauthor{}\def\yshift{-2ex}\def\quotefill{\hfill}}
   {\def\shadequoteauthor{\par\authorfill\shadedauthorformat{#2}}\def\yshift{2ex}}
\begin{snugshade}\begin{quote}\openquote}
{\shadequoteauthor\quotefill\closequote{\yshift}\end{quote}\end{snugshade}}

\begin{document}

\input{title}

\tableofcontents
\pagebreak

\listoffigures
\pagebreak

\pagebreak

% Neste trabalho prático os alunos deverão utilizar o "wireshark" para identificar a ocorrência de conexões e transferências de dados, envolvendo as camadas de aplicação, transporte e rede. Em telas capturas pelo "wireshark" os alunos deverão identificar os "pacotes" relacionados com criação da conexão, a transferência de dados e a liberação da conexão. Usar preferencialmente FTP e/ou HTTP. Fazer o relatório descrevendo as atividades.

\section{Configuração}

\begin{verbatim}
> meu ip: 192.168.1.103
> ip pronatec.mec.gov.br: 200.130.6.100
\end{verbatim}

\begin{verbatim}
> firefox -safe-mode -g # depois, ctrl + shift + p == mode privado
\end{verbatim}

\begin{verbatim}
> uname -a
> Linux p5qpl-am 3.8.0-19-generic #30-Ubuntu SMP Wed May 1 16:35:23 UTC 2013 x86_64 x86_64 x86_64 GNU/Linux
\end{verbatim}

\begin{verbatim}
> ifconfig -a
> wlan0 Link encap:Ethernet  HWaddr xx:xx:xx:xx:xx:xx
      inet addr:192.168.1.103  Bcast:192.168.1.255  Mask:255.255.255.0
      inet6 addr: fe80::a10:xxxx:fe5f:xxxx/64 Scope:Link
      UP BROADCAST RUNNING MULTICAST  MTU:1500  Metric:1
      RX packets:612485 errors:0 dropped:0 overruns:0 frame:0
      TX packets:412235 errors:0 dropped:0 overruns:0 carrier:0
      collisions:0 txqueuelen:1000
      RX bytes:870860278 (870.8 MB)  TX bytes:40905764 (40.9 MB)
\end{verbatim}

\pagebreak
\subsection{Significado das Cores na Tela de Captura do Wireshark}

\begin{figure}[ht]
\centering
% \includegraphics[width=\textwidth]{cores_ws}
\caption{Coloração Padrão do Wireshark}
\end{figure}

\pagebreak
\section{Liberação da conexão [FYN, ACK], [FYN]}
\begin{center}
\begin{figure}[ht]
% \includegraphics[width=\textwidth]{fynack}
\caption{Wireshark Filter: http.request}
\end{figure}
\end{center}

\pagebreak
\section{Criação da conexão [SYN], [SYN, ACK]}
\begin{center}
\begin{figure}[ht]
% \includegraphics[width=\textwidth]{synack}
\caption{Logo após acionar o site http://pronatec.mec.gov.br/}
\end{figure}
\end{center}

\pagebreak
\section{Transferência de dados pronatec.mec.gov.br}
\begin{center}
\begin{figure}[ht]
% \includegraphics[width=\textwidth]{http}
\caption{HTTP REQUEST http.request.method == GET}
\end{figure}
\end{center}

\pagebreak
\subsection{TCP STREAM [http.request.method == GET]}

\begin{huge}
   \href{Pronatec.mec.gov}{http://pronatec.mec.gov.br}
\end{huge}

\begin{verbatim}
GET / HTTP/1.1
Host: pronatec.mec.gov.br
User-Agent: Mozilla/5.0 (X11; Ubuntu; Linux x86_64; rv:25.0) Gecko/20100101 Firefox/25.0
Accept: text/html,application/xhtml+xml,application/xml;q=0.9,*/*;q=0.8
Accept-Language: en-US,en;q=0.5
Accept-Encoding: gzip, deflate
Cookie:
        PHPSESSID=4p433svpcr3c9d5jn31vou58e2;
        __utma=166947327.526842696.1385833606.1385833606.1385833606.1;
        __utmb=166947327.3.10.1385833606;
        __utmc=166947327;
        __utmz=166947327.1385833606.1.1.utmcsr=(direct)|utmccn=(direct)|utmcmd=(none)
Connection: keep-alive
If-Modified-Since: Sat, 30 Nov 2013 17:49:45 GMT
Cache-Control: max-age=0

HTTP/1.1 200 OK
Date: Sat, 30 Nov 2013 17:51:20 GMT
Server: Apache/2.2.14 (Ubuntu)
X-Powered-By: PHP/5.3.2-1ubuntu4.7
P3P: CP="NOI ADM DEV PSAi COM NAV OUR OTRo STP IND DEM"
Expires: Mon, 1 Jan 2001 00:00:00 GMT
Last-Modified: Sat, 30 Nov 2013 17:51:20 GMT
Cache-Control: post-check=0, pre-check=0
Pragma: no-cache
Vary: Accept-Encoding
Keep-Alive: timeout=10
Connection: Keep-Alive
Content-Type: text/html; charset=utf-8
Content-Encoding: gzip
Transfer-Encoding: chunked
\end{verbatim}

\pagebreak
\section{Conclusões tiradas sobre o HTTP tendo em vista o uso do wireshark e aprendizado teórico}

\subsection{Camada OSI:  Aplicação}

Quando eu escolho abrir uma página na web, atravéz no Firefox, um protocolo da camada de aplicação
chamado HTTP (Hypertext Transfer Protocol) \textbf{formata} e envia a requisição  do meu browser (um software de aplicação)
para o servidor. Assim como, também formata e envia a resposta devolta (response back) do servidor Web para o meu browser (client). \cite{rfc2916}

Suponha que eu escolha enxergar a página do Programa Nacional de Acesso ao Ensino Técnico e Emprego (Pronatec).
Eu digito http://pronatec.mec.gov.br no Firefox e clico Enter. Nesse instânte, a API do Firefox, transfere o meu requerimento
para protocolo HTTP. HTTP então permite que eu alcance o servidor Web deles.

Uma parte do método HTTP request (wireshark filter: http.request.method == GET) e diz para o servidor qual página eu quero acessar
e retirar informação dela.

Outra parte do HTTP request indica qual versão de HTTP que eu estou usando. Além de outras coisas.. Como se pode ver na imagem acima e na subseção Transferência de dados com a \href{http://pronatec.mec.gov.br/}{http://pronatec.mec.gov.br/}

Depois do recebimento do meu HTTP request, o servidor web deles responde para também para o meu
browser via HTTP. E a resposta deles inclui as informações que eu pedi anteriormente (informações = texto/imagens)
e que vão ser montadas na forma de uma página web, para que eu possa interagir com ela.

A versão HTTP usada, o tipo de resposta HTTP, e o comprimento da página. Todavia, se a página web é indisponível, inaucansável,
o servidor, pronatec.mec.gov.br, irá me enviar um resposta HTTP contendo uma mensagem de erro, como HTTP 404.

Perceba que a informação de um nó HTTP é desenhada para ser interpretada por outro nó HTTP. Via troca de protocólos. Todavia,
HTTP requests não podem atravesar a rede de um nó para outro, sem a assistência de camadas de mais baixo nível. \cite{osi}

\subsection{Camada OSI: Transporte}

Protocólos na camada de transporte aceitam informação da camada de sessão e (re)manejam
ela até o seu destino.

Isso significa que ela assegura que a informação irá ser transferida do nó A ao nó B (reliability), na sequência correta, e sem erros. Sem a camada de transporte, o pacote que vem da camada de sessão não poderiria ser verificados ou interpretados.

O protocolo da camada de transporte também assegura controle de fluxo, o qual é o processo de calibração apropriada da taxa de transmissão, baseada no quão rápido o remetente pode aceitar informação. Ou seja, existem muitos tipos de protocolos de transporte. Contudo redes modernas, como a Internet, abordam alguns tipos apenas.

Por exemplo: Quando baixamos uma página web, o protocolo da camada de transporte chamado TCP (Transmission Control Protocol) assegura com segurança a transmissão do protocolo de requisição http do cliente para o servidor ou o contrário. \cite{osi}

\subsection{Camada OSI: Rede}

A função princípal dos protocolos na camada de Rede, é \textbf{traduzir} os endereços da rede e decidir como rotear informação do remetende para o destinatário.

Essa camada implementa um sistema chamado endereçamento que identifica dispositivos na rede atravez de mapeamentos para valores numéricos.

Cada nó tem dois tipos de endereço. Um tipo de endereço é chamado de endereço de rede. Endereços de rede pertencem a uma hierarquia e pode ser visualizada pelos sistemas operacionais.

Os nós são hierarquicos pois eles contem sub-conjuntos de informações que determinam cota inferior e cota superior a esse nó.

O endereço da camada de rede é formatado diferente, dependendo a qual protocolo da camada de rede a rede está usando. Endereços de rede são também chamados de endereços lógicos ou endereços virtuais.

O segundo tipo de endereço dado a cada nó é chamado de endereço físico. \cite{osi}


\pagebreak
\begin{thebibliography}{unstr}

\bibitem {wireshark01} Wireshark Org., "Como Configurar Privilégios no Wireshark", Sítio: \href{http://wiki.wireshark.org/CaptureSetup/CapturePrivileges}
{http://wiki.wireshark.org/CaptureSetup/CapturePrivileges}; Acesso: \today

\bibitem {howto} Null-byte Wonder Howto, "How to Spy on Your "Buddys" Network Traffic: An Intro to Wireshark and the OSI Model". Sítio:
\href{http://null-byte.wonderhowto.com/how-to/spy-your-buddys-network-traffic-intro-wireshark-and-osi-model-0133807}
{http://null-byte.wonderhowto.com/how-to/spy-your-buddys-network-traffic-intro-wireshark-and-osi-model-0133807}; Acesso: \today

\bibitem {attack} Security.stackexchange.com, "HTTP 403 Response: Wireshark log analyzing". Sítio: \href{http://security.stackexchange.com/questions/12242/wireshark-log-analyzing}
{http://security.stackexchange.com/questions/12242/wireshark-log-analyzing}; Acesso: \today

\bibitem {rfc2916} Part of Hypertext Transfer Protocol -- HTTP/1.1 RFC 2616 Fielding, et al. Sítio: {http://www.w3.org/Protocols/rfc2616/rfc2616-sec10.html}{http://www.w3.org/Protocols/rfc2616/rfc2616-sec10.html}; Acesso: \today

\bibitem {osi} Networks Mania,  "OSI Model: 7 Layers". Sítio: \href{https://networksmania.wordpress.com/topics/osi-model/}{https://networksmania.wordpress.com/topics/osi-model/}; Acesso: \today

\end{thebibliography}

\end{document}
